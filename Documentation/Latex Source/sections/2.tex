\section{System Analysis}
		\vs
			\subsection{Software Requirement Specification (SRS)}
			\vs
			\hspace{1cm}The SRS is produced at the culmination of the analysis task. The function and performance allocated
			to software as a part of system engineering are refined by establishing a complete information
			description, a detailed functional and behavioral description, an indication of performance
			requirements and design constraints, appropriate validation criteria, and other data pertinent to
			requirements.
			\vs
			\large\textbf{Identification of need}
			\begin{itemize}
				\item Traditional conversion tool are often in the form of binaries complied to their specific platforms and therefore, there is not a single tool that can be used accross all platforms.
				\item Traditional video conversion solutions are often complex and hard to understand. Their user interface is full of an overwhelming amount of details and complexity.
				\item Traditional video conversion does not provide any cloud storage for videos. They can only be used offline and therefore, on slow computers such as handheld devices, there is no effecting and time friendly tool to convert videos.
				\item Traditional video conversion uses the client’s resources for video conversion.
				\item Traditional video conversion does not provide any concept of streaming videos and cloud storage.
			\end{itemize}
			\vs
			\large\textbf{Proposed Requirements}
				\vs
			\begin{itemize}
				\item The proposed system should be platform agnostic. The same application can be used on any device whether it's a powerful computer or a handheld device such as a mobile phone.
				
				\item The proposed system should be is easy to use and understand. The user interface should be designed in such a way that the user can easily understand the application. Necessary tooltips should also be provided wherever they are needed.
				\item The proposed system should provide cloud storage for videos and therefore, the videos once uploaded will be available to the user whenever they want. 
				\item The proposed system should use the server’s resources for video conversion and therefore, the videos can be converted quickly using a powerful server-side computer. Using this approach even large videos can be converted within minutes even on a small handheld device.
				\item The proposed system should also provide the facility of streaming service which enables the users to watch their videos once they are uploaded to the server.
			\end{itemize}
			\vs
			\large\textbf{Interface Requirements}
			\vs
			\begin{itemize}
				\item User Interface elements must be easy to understand.
				\item The user interface should be easy to learn. When the user uses the interface, they
				should know which element is used for which operation.
				\item The interface actions and elements should be consistent. When users press any
				button, required actions should be performed by the system along with
				the appropriate response to the user.
				\item The screen layout and accessibility should be clean and minimalistic.
				\item Understandability and vision of layout should be appealing and simple.
			\end{itemize}
		\vs
		\subsection{Risk Analysis}
		\vs
		\hspace{1cm}Uncertainty, which is constantly present in our daily lives, frequently impacts our decisions and actions. When we talk about risk, we normally mean the chance that some undesirable impact will occur. Hence, we normally seek to avoid or minimize risk. If there is a chance of rain, and we don't want to get wet, we may choose to stay indoors “avoiding that risk” or we may take an umbrella to minimize the impact of rain upon us. Uncertainty can impact our decisions and actions in desirable as well as undesirable ways. In risk analysis, we usually focus on what can go wrong“the outcomes that represent loss or damage” although an effective analysis will also help us understand what can go right as well. 
		\vs
		A risk assessment involves evaluating existing physical and environmental security and controls and accessing their adequacy relative to the potential threats of the organization. A system impact analysis involves identifying the critical system functions within the organization and determining the impact of not performing the business function beyond the maximum acceptable outage. Types of criteria that can be used to evaluate the impact include customer service, internal operations, legal/statutory, and financial.
		\vs
		A primary objective of business recovery planning is to protect the organization if all or part of its operations and/or computer services is rendered unusable. Each functional area of the organization should be analyzed to determine the potential risk and impact related to various disaster threats.
		\vs
		Regardless of the prevention techniques employed, possible threats that could arise inside or outside the organization need to be assessed. Although the exact nature of potential disasters or their resulting consequences is difficult to determine, it is beneficial to perform a comprehensive risk assessment of all threats that can realistically occur to the organization. Regardless of the type of threat, the goals of business recovery planning are to ensure the safety of users and the system. 
		\vs
		Uncertainty can arise in several ways:
		\begin{itemize}
			\item The user's video might not be supported by the server.
			\item The server might be busy in a lot of conversion requests and may not be able to accept new requests.
			\item Unavailability of the internet.
			\item Unavailability of the resources.
			\item Data corruption.
		\end{itemize}