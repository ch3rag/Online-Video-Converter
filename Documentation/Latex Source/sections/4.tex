\section{Feasibility Study}
		\vs
		\hspace{1cm} After understanding the project, study, and analyzing all the existing or required functionalities of the system, the next task is to do the feasibility study for the project. All projects are feasible – given unlimited resources and infinite time. This also includes consideration of all the possible ways to provide a solution to the given problem. The proposed solution should satisfy all the user requirements and should be flexible enough so that future changes can be easily done based on future upcoming requirements.
		\vs
		\subsection{Economic Feasibility}
		\vs
		\hspace{1cm}This is an important aspect to be considered while developing a project. We
		decided the technology based on the minimum possible cost factor.
		\begin{itemize}
			\item All hardware and software cost has to be borne by the organization.
			\item Overall we have estimated that the benefits the organization is going to
			receive from the proposed system will surely overcome and the later on
			running cost for system
		\end{itemize}
		Here, in \projectname; all elements are under defined economic constraints used for its development lie within the budget estimated for its development.
		\vs
		\subsection{Technical Feasibility}
		\vs
		\hspace{1cm}This included the study of function, performance, and constraints that may affect the ability to achieve an acceptable system. For this feasibility study, we studied complete functionality to be provided in the system, as described in the System Requirement Specification (SRS) and checked if everything was possible using a different type of frontend and backend platform.
		\vs
		For \projectname, following technical needs of the system may include:
		\begin{itemize}
			\item The facility to produce outputs in a given time. 
			\item Response time under certain conditions. 
			\item Ability to produce a certain volume of transactions at a particular speed. 
			\item Facility to communicate data to a distinct location. 
		\end{itemize}
		\vs
		Here, following hardware and software requirements must be fulfilled:
		\vs
					\bgroup
		\def\arraystretch{2}%
			\begin{tabular}{|m{5cm}|c|}
			\hline
			\textbf{Database Design} & Mongo DB v4.0.4+ \\
			\hline
			\textbf{Frontend Design} & Vue.js along with Vuetify and feathers-vuex \\
			\hline
			\textbf{Backend Design} & Node.js  along with express.js and feathers.js \\
			\hline
		\end{tabular}
	\egroup
		\vs
		Specific software and hardware products can then be evaluated keeping in view with the logical needs. The “\projectname” is technically feasible as all the software and hardware requirements are met by the organization.
		\vs
		\subsection{Operational Feasibility}
		\vs
		\hspace{1cm}No doubt the proposed system is fully GUI based which is very user friendly and all inputs to be taken all self-explanatory even to a layman. Besides, proper training has been conducted to let know the essence of the system to the users so that they feel comfortable with the new system. As far as our study is concerned the clients are comfortable and happy as the system has cut down their loads and doing.
		\vs
		 The “\projectname” is found to be feasible operationally because it is designed in such an interactive manner that users need not take any special training for operating the website.
		 \vs
		 \subsection{Cost Estimation Of The Project}
		 \vs
		 \hspace{1cm}Software cost comprises a small percentage of the overall computer-based system cost. There are several factors, which are considered, that can affect the ultimate cost of the software such as – human, technical, Hardware and Software availability, etc. The main point that was considered during the cost estimation of \textbf{\projectname}\space was its sizing. In spite of complete software sizing, function point and approximate lines of code were also used to size each element of the Software and their costing. The cost estimation is done by me for Project also depends upon the baseline metrics collected from past projects and these were used in conjunction with estimation variables to develop cost and effort projections.
		 \vs
		 We have basically estimated this project mainly on two bases:
		 \begin{itemize}
		 	\item 
		 	\textbf{\large Effort Estimation}
		 	
		 	This refers to the total man-hours required for the development of the project. It even includes the time required for doing documentation and user manual.
		 	\item 
		 	\textbf{\large Hardware Required}
		 	
		 	This includes the cost of the computer and the hardware cost required for the development of this project.		 	
		 \end{itemize}
	 Tools and methods used in the development of this project are free to use for noncommercial purposes. So, the development cost is negligible. Basic COCOMO\footnote{https://en.wikipedia.org/wiki/COCOMO}(Constructive Cost Mode) computes software development effort (and cost) as a function of program size. Program size is expressed in estimated thousands of source lines of code (SLOC\footnote{Source lines of code}, KLOC\footnote{Thousand lines of code}). 
	 \vs
	 COCOMO applies to three classes of software projects: 
	 \begin{itemize}
	 	\item 
	 	\textbf{\large Organic projects}
	 	
	 	Small teams with good experience working with less than rigid requirements. 

	 	
	 	\item 
	 	\textbf{\large Semi-Detached Projects}
	 	
	 	Moderate sized teams with mixed experience working with a mixture of rigid and less than rigid requirements.
	 	
	 	\item 
	 	\textbf{\large Embedded Projects}
	 	
	 	Developed within a set of {\em tight} constraints. It is also combination of organic and semi-detached projects. 
	 	
	 \end{itemize}
	 \vs
	 The basic COCOMO equations takes the form:
	\[\textrm{Effort Applied}(E) = a_b.(KLOC).b_b  \textrm{\space [man-month]} \]
	\[\textrm{Development Time}(D) = c_b.(\textrm{Effort Applied}).d_b  \textrm{\space [months]} \]
	\[\textrm{People Required}(P) = \textrm{Effort Applied / Development Time}\]
	\vs
	Where, KLOC is the estimated number of delivered lines (expressed in thousands) of
	code for project. The coefficients a\textsubscript{b}, b\textsubscript{b}, c\textsubscript{b} and d\textsubscript{b} are given in the following table\footnote{The values listed here are from the original analysis, with a modern reanalysis producing different values.}.
	\vs
	Basic COCOMO is good for a quick estimate of software costs. However, it does not
	account for differences in hardware constraints, personnel quality, and experience, use of
	modern tools and techniques, and so on. 
	\vs
	\textbf{\large Estimating The Cost Of The Project}
	\vs
	Since this is an organic project, we’ll take the respective values for the coeffecients:
	\[\textrm{Lines Of Code}(LOC) = 5042\]
	Therefore, 
		\[KLOC = 5042 / 1000\]
		\[KLOC = 5.042      \]
	Taking the values of the coeffecients as:
	\vs
				\bgroup
	\def\arraystretch{2}%
	\begin{center}
		\begin{tabular}{|c|c|c|c|}
				\hline
				a\textsubscript{b} & a\textsubscript{b} &  a\textsubscript{b} & a\textsubscript{b} \\
				\hline
				2.4 & 1.05 & 2.5 & 0.38 \\
				\hline
		\end{tabular}
	\egroup
	\end{center}
	\vs
	\begin{itemize}
		\item \textbf{Effort Applied}
		
			\[\textrm{Effort Applied}(E) = a_b.(KLOC).b_b  \textrm{\space [man-month]} \]
			\[E = 2.4(5.042)^{1.05} \]
			\[E = 2.4(5.546679) \]
			\textbf{\[E = 13.120313 \textrm{\space [man-month]}\]}
			
		\item \textbf{Development Time}
			
	\[\textrm{Development Time}(D) = c_b.(\textrm{Effort Applied}).d_b  \textrm{\space [months]} \]
			\[D = 2.5(13.120312)^{0.38} \]
			\[D = 2.5(2.65961) \]
			\textbf{\[D = 6.65 \textrm{\space [months]}\]}
			
					\item \textbf{People Required}
			
	\[\textrm{People Required}(P) = \textrm{Effort Applied / Development Time}\]
			\[P = 13.120313 / 6.65 \]
			\[P \approx 2\]
	\end{itemize}